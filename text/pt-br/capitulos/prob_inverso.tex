\chapter{Formulação matemática do problema inverso}

Dado um conjunto de $N$ observações, feitas em diferentes posições, tempos, etc.,
de\-fi\-ni\-mos um {\it vetor de dados observados}

\begin{equation}
\vect{d}^{\thinspace o} =
    \begin{bmatrix}
    d_1^{\thinspace o} \\
    d_2^{\thinspace o} \\
    \vdots \\
    d_N^{\thinspace o}
    \end{bmatrix},
\end{equation}

\noindent em que $d_i^{\thinspace o}$, $i = 1, 2, 3, \dotsc, N$, é o dado
observado na $i$-ésima posição, tempo, etc.
De forma análoga, definimos um {\it vetor de dados preditos}

\begin{equation}
\vect{d}^{\thinspace p} =
    \begin{bmatrix}
    d_1^{\thinspace p} \\
    d_2^{\thinspace p} \\
    \vdots \\
    d_N^{\thinspace p}
    \end{bmatrix},
\end{equation}

\noindent em que $d_i^{\thinspace p}$, $i = 1, 2, 3, \dotsc, N$, é o dado predito
calculado na mesma posição, tempo, etc., que o $i$-ésimo dado observado.
Continuando no espírito de definição de vetores, definimos também um vetor que
agrupa todos os $M$ parâmetros, denominado {\it vetor de parâmetros}

\begin{equation}
\vect{p} =
    \begin{bmatrix}
    p_1 \\
    p_2 \\
    \vdots \\
    p_M
    \end{bmatrix},
\label{eq:param_vect}
\end{equation}

\noindent em que $p_j$, $j = 1, 2, 3, \dotsc, M$, é o $j$-ésimo parâmetro.
\\
\indent Como vimos no Capítulo \ref{chap:intro}, os dados preditos são descritos
por uma função dos parâmetros, ou seja,

\begin{equation}
d^{\thinspace p}_i = f_i(\vect{p})\thinspace.
\label{eq:fi}
\end{equation}

\noindent Desta forma, podemos dizer que o vetor de dados preditos é uma função
dos parâmetros

\begin{equation}
\vect{d}^{\thinspace p}= \vect{f}(\vect{p}) =
    \begin{bmatrix}
    f_1(\vect{p}) \\
    f_2(\vect{p}) \\
    \vdots \\
    f_N(\vect{p})
    \end{bmatrix}.
\label{eq:dados_preditos}
\end{equation}

\indent O problema inverso consiste em encontrar um vetor de parâmetros $\vect{p}$
que produza os dados preditos mais próximos possívies dos dados observados.
Para determinar a ``proximidade'' entre os dados observados e os dados preditos,
é necessário quantificar a distância entre eles.
Isto é feito em termos da norma do {\it vetor de resíduos}

\begin{equation}
\vect{r} = \vect{d}^{\thinspace o} - \vect{f}(\vect{p}),
\end{equation}

\noindent em que $\vect{d}^{\thinspace o}$ é o vetor de {\it dados observados}
e $\vect{f}(\vect{p})$ é o vetor de {\it dados preditos}.
\\
\indent Para quantificar a distância entre os dados observados e os dados
preditos utiliza-se, usualmente, o quadrado da
norma quadrática (também conhecida como norma $\ell_2$ ou norma Euclidiana)
do vetor de resíduos

\begin{equation}
\norm{\vect{r}}_2^2 =
    \sum\limits_{i=1}^N \left[d^{\thinspace o}_i - f_i(\vect{p})\right]^2 \, .
\label{eq:norma_l2}
\end{equation}

\noindent Esta equação é também uma função escalar
dos parâmetros. Assim sendo, definimos uma função $\phi(\vect{p})$, chamada de
{\it função do ajuste}, como

\begin{equation}
\phi(\vect{p}) = \norm{\vect{r}}_2^2 =
    \sum\limits_{i=1}^N \left[d^{\thinspace o}_i - f_i(\vect{p})\right]^2 \, .
\label{eq:ajuste_sum}
\end{equation}

\indent Lembramos que o quadrado da norma Euclidiana de um vetor é igual ao
produto escalar do vetor com ele mesmo, ou seja,

\begin{equation}
\norm{\vect{r}}_2^2 = \vect{r} \cdot \vect{r} =
    r_1r_1 + r_2r_2 + \dotsb + r_Nr_N \thinspace .
\label{eq:dotprod}
\end{equation}

\noindent Como o vetor $\vect{r}$ é um {\it vetor coluna} (matriz com uma
coluna), podemos escrever o produto escalar da equação \ref{eq:dotprod} como

\begin{equation}
\vect{r} \cdot \vect{r} = \vect{r}^T \vect{r} =
    \begin{bmatrix}
        r_1 & r_2 & \ldots & r_N
    \end{bmatrix}
    \begin{bmatrix}
        r_1 \\ r_2 \\ \vdots \\ r_N
    \end{bmatrix} .
\label{eq:vectdotprod}
\end{equation}

\noindent Dessa forma podemos reescrever a função do ajuste (equação
\ref{eq:ajuste_sum}) como

\begin{equation}
\phi(\vect{p}) = \vect{r}^T \vect{r} =
    \left[\vect{d}^{\thinspace o} - \vect{f}(\vect{p})\right]^T
    \left[\vect{d}^{\thinspace o} - \vect{f}(\vect{p})\right] .
\label{eq:ajuste}
\end{equation}

\indent Neste ponto é importante ressaltar o seguinte conceito:

\begin{quote}
{\tt A {\bf função do ajuste} é uma função escalar que
quantifica a dis\-tân\-cia entre os dados observados e os dados preditos para um
de\-ter\-mi\-na\-do vetor de parâmetros $\vect{p}$.}
\end{quote}

\indent Perante este conceito, o problema inverso consiste em determinar um
vetor $\opt{p}$ que minimiza a função do ajuste $\phi(\vect{p})$.
Matematicamente, isso equivale a encontrar o vetor $\opt{p}$ tal que o gradiente
da função $\phi(\vect{p})$ avaliado em $\opt{p}$ seja igual ao vetor nulo.
Este vetor $\opt{p}$ é um {\it ponto extremo} da função
$\phi(\vect{p})$.
\\
\indent O gradiente da função $\phi(\vect{p})$, avaliado em um $\vect{p}$ qualquer,
é um vetor $M$-dimensional definido como (ver Apêndice \ref{chap:opmat})

\begin{equation}
\vect{\nabla} \phi(\vect{p}) =
    \begin{bmatrix}
        \dfrac{\partial \phi(\vect{p})}{\partial p_1} \vspace{0.3cm}\\
        \dfrac{\partial \phi(\vect{p})}{\partial p_2}\\
        \vdots \\
        \dfrac{\partial \phi(\vect{p})}{\partial p_M}
    \end{bmatrix} ,
\label{eq:gradphi_partial}
\end{equation}

\noindent em que $\vect{\nabla}$ é o operador gradiente (Apêndice \ref{chap:opmat}).
A partir da equação \ref{eq:ajuste}, a expressão para o $i$-ésimo elemento do
vetor gradiente é

\begin{equation}
\begin{split}
\dfrac{\partial \phi(\vect{p})}{\partial p_i} &=
    \dfrac{\partial}{\partial p_i}\left\{
        \left[\vect{d}^{\thinspace o} - \vect{f}(\vect{p})\right]^T
        \left[\vect{d}^{\thinspace o} - \vect{f}(\vect{p})\right]
    \right\} \\[0.5cm]
    &=
    \underbrace{
    \left\{-\dfrac{\partial\vect{f}(\vect{p})}{\partial p_i}^T
            \left[\vect{d}^{\thinspace o} - \vect{f}(\vect{p})\right]
    \right\}}_{\text{escalar}}
    +
    \underbrace{
    \left\{-\left[\vect{d}^{\thinspace o} - \vect{f}(\vect{p})\right]^T
            \dfrac{\partial\vect{f}(\vect{p})}{\partial p_i}
    \right\}}_{\text{escalar}}
\end{split} .
\label{eq:del_phi_del_pi}
\end{equation}

\indent Lembrando que o transposto de um escalar é igual a ele mesmo, podemos
tomar o transposto do segundo termo do lado direito da equação
\ref{eq:del_phi_del_pi}, obtendo

\begin{equation}
\dfrac{\partial \phi(\vect{p})}{\partial p_i} =
    -2\dfrac{\partial\vect{f}(\vect{p})}{\partial p_i}^T
    \left[\vect{d}^{\thinspace o} - \vect{f}(\vect{p})\right],
\label{eq:del_phi_del_pi_simple}
\end{equation}

\noindent em que $\dfrac{\partial\vect{f}(\vect{p})}{\partial p_i}$ é um vetor
$N$-dimensional (ver Apêndice \ref{chap:opmat}) dado por

\begin{equation}
\dfrac{\partial\vect{f}(\vect{p})}{\partial p_i} =
    \begin{bmatrix}
        \dfrac{\partial f_1(\vect{p})}{\partial p_i} \vspace{0.3cm}\\
        \dfrac{\partial f_2(\vect{p})}{\partial p_i}\\
        \vdots \\
        \dfrac{\partial f_N(\vect{p})}{\partial p_i}
    \end{bmatrix}.
\label{eq:del_f_del_pi}
\end{equation}

\indent Substituindo as equações \ref{eq:del_phi_del_pi_simple} e
\ref{eq:del_f_del_pi} na expressão do gradiente (equação \ref{eq:gradphi_partial})
obtemos

\begin{equation}
\begin{split}
\vect{\nabla} \phi(\vect{p}) &=
        \begin{bmatrix}
            -2\dfrac{\partial\vect{f}(\vect{p})}{\partial p_1}^T
                \left[\vect{d}^{\thinspace o} - \vect{f}(\vect{p})\right]
                \vspace{0.3cm} \\
            -2\dfrac{\partial\vect{f}(\vect{p})}{\partial p_2}^T
                \left[\vect{d}^{\thinspace o} - \vect{f}(\vect{p})\right]\\
            \vdots \\
            -2\dfrac{\partial\vect{f}(\vect{p})}{\partial p_M}^T
                \left[\vect{d}^{\thinspace o} - \vect{f}(\vect{p})\right]
        \end{bmatrix}
    \\[0.5cm] &=
        -2
        \begin{bmatrix}
            \dfrac{\partial\vect{f}(\vect{p})}{\partial p_1}^T\vspace{0.3cm} \\
            \dfrac{\partial\vect{f}(\vect{p})}{\partial p_2}^T \\
            \vdots \\
            \dfrac{\partial\vect{f}(\vect{p})}{\partial p_M}^T
        \end{bmatrix}
        \left[\vect{d}^{\thinspace o} - \vect{f}(\vect{p})\right].
\end{split}
\label{eq:gradphi_partial_f}
\end{equation}

\noindent Por fim, o gradiente da função do ajuste $\phi(\vect{p})$ pode ser
escrito como

\begin{equation}
\vect{\nabla} \phi(\vect{p}) = -2\mat{G}(\vect{p})^T
    \left[\vect{d}^{\thinspace o} - \vect{f}(\vect{p})\right].
\label{eq:gradphi}
\end{equation}

\noindent em que

\begin{equation}
\mat{G}(\vect{p}) =
\begin{bmatrix}
    \dfrac{\partial\vect{f}(\vect{p})}{\partial p_1} &
    \dfrac{\partial\vect{f}(\vect{p})}{\partial p_2} &
    \ldots &
    \dfrac{\partial\vect{f}(\vect{p})}{\partial p_M}
\end{bmatrix}
=
\begin{bmatrix}
    \dfrac{\partial f_1(\vect{p})}{\partial p_1} &
        \dfrac{\partial f_1(\vect{p})}{\partial p_2} &
        \ldots &
        \dfrac{\partial f_1(\vect{p})}{\partial p_M}
    \vspace{0.3cm}\\
    \dfrac{\partial f_2(\vect{p})}{\partial p_1} &
        \dfrac{\partial f_2(\vect{p})}{\partial p_2} &
        \ldots &
        \dfrac{\partial f_2(\vect{p})}{\partial p_M}
    \\
    \vdots & \vdots & \ddots & \vdots
    \\
    \dfrac{\partial f_N(\vect{p})}{\partial p_1} &
        \dfrac{\partial f_N(\vect{p})}{\partial p_2} &
        \ldots &
        \dfrac{\partial f_N(\vect{p})}{\partial p_M}
\end{bmatrix}.
\label{eq:jacobian}
\end{equation}

\noindent A matriz $\mat{G}(\vect{p})$ de dimensão $N \times M$ é a
{\it matriz Jacobiana} de $\vect{f}(\vect{p})$.

\begin{quote}
{\tt Em problemas inversos, essa matriz $\mat{G}$ é comumente denominada
{\bf matriz de sensibilidade}, uma vez que o $i$-ésimo elemento de sua $j$-ésima
coluna expressa a sensibilidade do $i$-ésimo dado pre\-di\-to em re\-la\-ção à variações
do $j$-ésimo parâmetro.}
\end{quote}

\indent As equações \ref{eq:gradphi} e \ref{eq:jacobian} mostram que o gradiente
da função do ajuste $\phi(\vect{p})$ depende do vetor de dados preditos
$\vect{f}(\vect{p})$ (equação \ref{eq:dados_preditos}) e sua derivada em relação
aos parâmetros $\vect{p}$.
Sendo assim, a função $f$ que relaciona os parâmetros aos dados preditos determina
o comportamento do gradiente de $\phi(\vect{p})$.
Nas próximas seções analisaremos os casos em que a função $f$ é linear ou
não-linear em relação aos parâmetros.
Essa análise nos permitirá compreender a influência da função $f$ na busca pelo
vetor de parâmetros $\opt{p}$ que minimiza a função $\phi(\vect{p})$.

