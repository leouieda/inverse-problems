\chapter{Procedimentos práticos}
\label{chap:proc_praticos}

Durante a solução de problemas inversos nos deparamos com diversos obstáculos, 
como:

\begin{itemize}
\item Por que a inversão não foi capaz de ajustar os dados observados?
\item Como determinar um valor adequado para o parâmetro de regularização?
\item Como analisar a estabilidade da solução?
\end{itemize}

\noindent Neste capítulo descreveremos alguns procedimentos práticos para abordar
estas questões.

\section{Ajustar os dados observados}

Em situações reais, nem sempre é possível ajustar perfeitamente os dados observados.
Isso pode acontecer por três principais razões:

\begin{enumerate}
\item Os dados observados contêm ruído;
\item A função $f_i(\vect{p})$ escolhida para descrever a relação entre os
parâmetros e os dados preditos não é capaz de descrever o sistema físico
de forma satisfatória;
\item Excesso de regularização;
\end{enumerate}

\indent O primeiro caso é esperado pois observações sempre contêm algum tipo de
ruído.
Embora existam técnicas para a atenuação de ruído, é impossível que ele seja
removido completamente.
Além disso, ruídos de caráter aleatório não são descritos pela relação funcional
entre os parâmetros e os dados preditos (equação \ref{eq:fi}).
Sendo assim, não ajustar perfeitamente os dados observados é aceitável, desde que
o desajuste esteja dentro do nível de ruído estimado dos dados.
\\
\indent Já no segundo caso, a função escolhida para descrever a relação entre os
pa\-râ\-me\-tros e os dados preditos não representa o sistema físico de forma adequada.
Nesse caso, mesmo se os dados observados fossem isentos de ruído (situação
impossível em condições reais), não seria possível estimar parâmetros que
ajustassem os dados observados. 
Isso é um indício de que a hipótese escolhida para representar o sistema físico
não é válida.
É crucial ressaltar que este é um resultado tão importante quanto encontrar uma
hipótese que descreva o sistema físico de forma satisfatória.
Este tipo de resultado nos permite descartar hipóteses (i.e., cenários geológicos)
{\it com base nos dados geofísicos}.
\\
\indent O terceiro caso, em que há excesso de regularização, é devido a utilização
de valores demasiado elevados do {\it parâmetro de regularização} $\mu$
(equação \ref{eq:objetivo}).
Valores elevados de $\mu$ dão excessiva importância para a informação a priori
(i.e., regularização), o que causa um desajuste dos dados.
Já valores baixos de $\mu$ priorizam o ajuste dos dados e negligenciam a informação
a priori. Neste caso, embora haja um bom ajuste dos dados, o problema 
inverso se torna instável. Na Seção \ref{sec:valor_mu} trataremos da escolha de uma
valor adequado para $\mu$.

\section{Determinar um valor para o parâmetro de regularização}
\label{sec:valor_mu}

O valor do parâmetro de regularização $\mu$ (equação \ref{eq:objetivo}) {\bf não}
pode ser determinado para um caso geral.
O valor de $\mu$ pode ser diferente para cada dado observado e cada parametrização
utilizada.
Além disto, no caso em que haja mais de um função regularizadora, pode ser
desejável impor alguma delas mais fortemente que as outras.
Por exemplo, se utilizarmos ambas as funções regularizadoras de
suavidade e igualdade (equações \ref{eq:suavidade} e \ref{eq:igualdade},
respectivamente),
podemos desejar que a igualdade seja imposta mais que a
suavidade. No entanto, há procedimentos que permitem encontrar valores de $\mu$
que sejam considerados adequados. 
\\
\indent Um valor adequado de $\mu$ deve proporcionar um equilíbrio entre
a estabilidade da solução e o ajuste dos dados observados.
Embora existam alguns métodos para escolha automática do parâmetro de regularização
\citep{aster}, ainda não há uma maneira de determinar um valor ótimo.
Além disso, estes métodos automáticos são aplicáveis somente a problemas inversos
que utilizam uma única função regularizadora.
Sendo assim, apresentamos a seguir um procedimento prático que consiste em escolher o
menor valor possível para o parâmetro de regularização para que o problema
inverso seja bem-posto (i.e., com solução única e estável).
\\
\indent Seja um conjunto de $N$ dados observados e $M$ parâmetros,
o procedimento prático segue as seguintes etapas:

\begin{enumerate}
\item Gerar um vetor $\vect{e}$ de $N$ valores aleatórios.
Cada valor deve ser a realização de uma variável aleatória com distribuição
Gaussiana de média zero e desvio padrão igual ao nível de ruído estabelecido
para os dados;
\item Gerar um vetor $\vect{d}\thinspace'$ de {\it dados observados perturbados} a partir
da soma do vetor de valores aleatórios $\vect{e}$ e o vetor de dados observados
$\vect{d}^{\thinspace 0}$
\[
\vect{d}\thinspace' = \vect{d}^{\thinspace 0} + \vect{e} \thinspace ;
\]
\item Repetir as etapas 1 e 2 para gerar um conjunto de $Q$ vetores de dados
observados perturbados diferentes;
\item Estabelecer um valor pequeno para o parâmetro de regularização $\mu$;
\item Utilizar este valor de $\mu$ em $Q$ inversões para estimar $Q$ vetores de
parâmetros diferentes. Cada vetor de parâmetros corresponde a um dos vetores de
dados observados perturbados;
\item Calcular o desvio padrão dos $Q$ valores obtidos para o $j$-ésimo parâmetro;
\item Repetir a etapa 6 para cada um dos $M$ parâmetros;
\item Se ao menos um dos $M$ parâmetros apresentar um desvio padrão maior que um
valor máximo preestabelecido, considere que o valor de $\mu$ não é adequado.
Nesse caso, aumente ligeiramente o valor de $\mu$ e repita as etapas 5-7;
\item Se nenhum dos $M$ parâmetros apresentar um desvio padrão maior que um
valor máximo preestabelecido, considere que o valor de $\mu$ é adequado e o
problema inverso está bem regularizado;
\end{enumerate}

\noindent As etapas 1-3 e 5-8 acima são o procedimento adotado para a
{\it análise da estabilidade} da solução.
